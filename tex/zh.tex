%------------------------
% Ethan's Résumé Template (XeLaTeX Chinese Version)
%------------------------
\documentclass[letterpaper,11pt]{article}

% 基础包
\usepackage{fontspec}
\usepackage{xeCJK}
\usepackage{latexsym}
\usepackage[empty]{fullpage}
\usepackage{titlesec}
\usepackage{bm}
\usepackage{marvosym}
\usepackage{xcolor}
\usepackage{verbatim}
\usepackage{enumitem}
\usepackage[hidelinks]{hyperref}
\usepackage{fancyhdr}
\usepackage{tabularx}
\usepackage{fontawesome5}
\usepackage{ragged2e}
\usepackage{etoolbox}
\usepackage{tikz}
\usepackage{hyperref}

% ================= 字体设置 =================

\linespread{1.05}

% ================= 页面布局 =================
\pagestyle{fancy}
\fancyhf{}
\fancyfoot{}
\renewcommand{\headrulewidth}{0pt}
\renewcommand{\footrulewidth}{0pt}

% 调整页边距
\addtolength{\oddsidemargin}{-0.5in}
\addtolength{\evensidemargin}{-0.5in}
\addtolength{\textwidth}{1in}
\addtolength{\topmargin}{-.5in}
\addtolength{\textheight}{1.0in}

\urlstyle{same}

\raggedbottom
\raggedright
\setlength{\tabcolsep}{0in}
\setlength{\footskip}{5pt}

% ================= Section 样式 =================
\titleformat{\section}{
  \vspace{-2pt}\large\bfseries
}{}{0em}{}[\color{black}\titlerule\vspace{-5pt}]

\titleformat{\subsection}{
  \vspace{-2pt}\normalsize\bfseries
}{}{0em}{}[\color{black}\titlerule\vspace{-5pt}]

% ================= 自定义命令 =================
\newcommand{\cvitem}[1]{
  \item\small{{#1\vspace{-2pt}}}
}

\newcommand{\cvheading}[4]{
  \vspace{-2pt}
  \item
  \begin{tabular*}{\textwidth}{l@{\extracolsep{\fill}}r}
    \textbf{#1} & #2 \\
    \small#3 & \small#4 \\
  \end{tabular*}
  \vspace{-7pt}
}

\newcommand{\twopartheader}[2]{
  \vspace{-2pt}
  \noindent
  \begin{tabular*}{\textwidth}{l@{\extracolsep{\fill}}r}
    #1 & #2 \\
  \end{tabular*}
  \vspace{-7pt}
}

\newcommand{\cvheadingstart}{\begin{itemize}[leftmargin=0in, label={}]}
\newcommand{\cvheadingend}{\end{itemize}}
\newcommand{\cvitemstart}{\begin{itemize}[label=\textopenbullet]\justifying}
\newcommand{\cvitemend}{\end{itemize}\vspace{-5pt}}

\renewcommand\labelitemii{$\vcenter{\hbox{\footnotesize$\bullet$}}$}

% ===================== 正文开始 ======================
\begin{document}

% 联系方式
\begin{center}
  {\LARGE\bfseries 杨希渊} \\
  \vspace{2pt}
  \href{mailto:yangxiyuan@sjtu.edu.cn}{yangxiyuan@sjtu.edu.cn}
  $\ \diamond\ $
  上海交通大学
  $\ \diamond\ $
  上海,中国
  \\
  GitHub: \href{https://github.com/xiyuanyang-code}{https://github.com/xiyuanyang-code}
  \\
  简历主页:\href{https://xiyuanyang-code.github.io/cv}{https://xiyuanyang-code.github.io/cv/}
\end{center}

% ================= 教育 =================
\section{教育背景}
\cvheadingstart

\cvheading
{上海交通大学}{上海,中国}
{人工智能学院·本科}{2024.09 -- 至今}

GPA: 4.14/4.3(专业排名:\textbf{1/62})

成绩:94.0/100

奖学金:国家奖学金(前 3\%)、致远荣誉奖学金(前 5\%)

\begin{itemize}[nosep]
  \item 编程综合实践:100/100
  \item 线性代数(荣誉):98/100
  \item 程序设计思想与方法(荣誉):98/100
\end{itemize}

共 \textbf{17} 门课程获得 A/A+,其中 \textbf{9 门为 A+},涵盖全部专业核心课程。

\cvheadingend

% ================= 技术项目 =================
\section{技术项目}

\textbf{AlphaBuild:基于强化学习的因子挖掘系统}

构建基于 PPO 的因子生成方法,改进 \href{https://github.com/RL-MLDM/alphagen}{AlphaGen} 框架,通过并行因子池与 PCA 选择增强回测表现。

\medskip

% todo add more technical projects

\textbf{ProbeCode:基于 MCP 的智能编程 Agent}

\url{https://github.com/xiyuanyang-code/ProbeCode}

基于 MCP 框架搭建编程智能体,优化编程智能体长上下文记忆能力,实现对大型项目级代码库的精准理解。

\medskip

\textbf{技术博客}

\url{https://xiyuanyang-code.github.io}

累计发表 \textbf{120+} 篇文章,总字数 \textbf{48w+}。同时作为 GitHub 活跃贡献者,提交次数超 1700 余次。


% ================= 科研 =================
\section{科研经历}

研究方向:\textbf{自主智能体、群体智能}

本科期间于 \textbf{MAGIC 实验室}(导师:陈思衡)参与本科生科研实习工作,并发表多篇论文。

\medskip

\textbf{AppCopilot:移动端智能体系统}

\url{https://arxiv.org/abs/2509.02444}  
\url{https://github.com/OpenBMB/AppCopilot}(\textbf{600+ stars})

% todo optimize chinese version abstract
面向当前多模态终端智能体“看不准、想不远、做不快”的三大痛点,在感知、规划、执行三层构建多模态多智能体端侧助理 AppCopilot,覆盖数据标注、模型训推、APP研发的全链路。

\medskip

\textbf{InfoMosaic-Bench:工具增强信息检索智能体基准}

\url{https://arxiv.org/abs/2510.02271}  
\url{https://github.com/DorothyDUUU/Info-Mosaic}(\textbf{100+ stars})

InfoMosaic-Bench 系首个专为\textbf{工具增强智能体}设计的多源信息检索基准,涵盖医学、金融、地图、视频、网页及跨领域整合六大领域。针对现有智能体过度依赖噪声化网络搜索且缺乏精准领域知识之弊,引入模型上下文协议对接数千专用工具,评估其与通用搜索协同解决复杂任务之能力。



% ================= 技能 =================
\newpage
\section{技能}

\begin{enumerate}[itemsep=2pt, topsep=2pt, parsep=0pt, partopsep=0pt]
    \item \textbf{编程语言}:Python(熟练)、C++(熟练)、Rust、JavaScript、HTML、CSS  
    \item \textbf{工具链}:Git、LaTeX、Shell、Docker 
    \item \textbf{Python 库}:Torch(熟练)、Numpy、Pandas、Matplotlib、BeautifulSoup、FastMCP  
    \item \textbf{机器学习、深度学习}:掌握深度学习模型网络架构与训练方法  
    \item \textbf{大模型与智能体}:熟悉 LLM 架构、后训练增强、多智能体系统 
    \item \textbf{语言能力}:中文(母语)、英语(CET-6: 584)
\end{enumerate}


% ================= 奖项 =================
\section{奖项与荣誉}

\textbf{2025 挑战杯 “揭榜挂帅” 赛道}

挑战杯 “揭榜挂帅” 人工智能领域 多智能体赛道 擂主(特等奖第一名)

\medskip

\textbf{2025 本科生国家奖学金 (2/62)}

\medskip

\textbf{2025 MCM/ICM 美国大学生数学建模竞赛 M 奖}

\medskip

\textbf{2024 致远荣誉奖学金 (前 5\%)}


\end{document}
