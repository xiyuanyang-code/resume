%------------------------
% Ethan's Résumé Template
% Author: necusjz
% License: CC-BY-4.0
%------------------------
\documentclass[letterpaper,11pt]{article}

\usepackage{latexsym}
\usepackage[empty]{fullpage}
\usepackage{titlesec}
\usepackage{bm}
\usepackage{marvosym}
\usepackage[usenames,dvipsnames]{color}
\usepackage{verbatim}
\usepackage{enumitem}
\usepackage[hidelinks]{hyperref}
\usepackage{fancyhdr}
\usepackage[english]{babel}
\usepackage{tabularx}
\usepackage{fontawesome5}
\usepackage{ragged2e}
\usepackage{etoolbox}
\usepackage{tikz}
\usepackage{hyperref}
\definecolor{lightblue}{RGB}{173, 216, 230} 
\hypersetup{
    colorlinks = true,  
    urlcolor   = blue,  
}
\input{glyphtounicode}

% font options
\usepackage{newpxtext}
\linespread{1.05}  % palladio needs more leading (space between lines)
\usepackage[T1]{fontenc}

\pagestyle{fancy}
\fancyhf{}  % clear all header and footer fields
\fancyfoot{}
\renewcommand{\headrulewidth}{0pt}
\renewcommand{\footrulewidth}{0pt}

% adjust margins
\addtolength{\oddsidemargin}{-0.5in}
\addtolength{\evensidemargin}{-0.5in}
\addtolength{\textwidth}{1in}
\addtolength{\topmargin}{-.5in}
\addtolength{\textheight}{1.0in}

\urlstyle{same}

\raggedbottom
\raggedright
\setlength{\tabcolsep}{0in}
\setlength{\footskip}{5pt}

% sections formatting
\titleformat{\section}{
  \vspace{-2pt}\scshape\raggedright\large
}{}{0em}{}[\color{black}\titlerule\vspace{-5pt}]


% subsection formatting
\titleformat{\subsection}{
  \vspace{-2pt}\scshape\raggedright\normalsize
}{}{0em}{}[\color{black}\titlerule\vspace{-5pt}]
% ensure that generate pdf is machine readable/ATS parsable
\pdfgentounicode=1

% custom commands
\newcommand{\cvitem}[1]{
  \item\small{
    {#1\vspace{-2pt}}
  }
}

\newcommand{\cvheading}[4]{
  \vspace{-2pt}\item
    \begin{tabular*}{\textwidth}[t]{l@{\extracolsep{\fill}}r}
      \textbf{#1} & #2 \\
      \small#3 & \small #4 \\
    \end{tabular*}\vspace{-7pt}
}

\newcommand{\twopartheader}[2]{
  \vspace{-2pt}\noindent
  \begin{tabular*}{\textwidth}{l@{\extracolsep{\fill}}r}
    #1 & #2 \\
  \end{tabular*}\vspace{-7pt}
}

\newcommand{\cvheadingstart}{\begin{itemize}[leftmargin=0in, label={}]}
\newcommand{\cvheadingend}{\end{itemize}}
\newcommand{\cvitemstart}{\begin{itemize}[label=\textopenbullet]\justifying}
\newcommand{\cvitemend}{\end{itemize}\vspace{-5pt}}

\newcommand{\cvskill}[2]{
  \textcolor{black}{\textbf{#1}}\hfill
  \foreach \x in {1,...,5}{%
    \space{\ifnumgreater{\x}{#2}{\color{black!80!white!20}}{\color{black}}\faSquare}}\par%
  \vspace{-2pt}
}

\renewcommand\labelitemii{$\vcenter{\hbox{\footnotesize$\bullet$}}$}


% =======================begin==============================
\begin{document}

% contact information
\begin{center}
  \textbf{\LARGE\scshape Xiyuan Yang} \\
  \vspace{1pt}\small
  \href{mailto:}{yangxiyuan@sjtu.edu.cn}
  $\ \diamond\ $
  Shanghai Jiao Tong University
  $\ \diamond\ $
  Shanghai, China
  \\
  Github: \href{https://github.com/xiyuanyang-code}{xiyuanyang-code}
  % $\ \diamond\ $
  \\
  Online Resume: \href{https://xiyuanyang-code.github.io/resume}{https://xiyuanyang-code.github.io/resume/}
\end{center}



\section{Education}
\cvheadingstart
  \cvheading
    {Shanghai Jiao Tong University}{Shanghai, China}
    {Bachelor's degree, School of artificial intelligence}{09/2024 - present (Still freshman year)}

    GPA: 4.12/4.3 \textbf{(Ranked 1 out of 62)}

    Score: 93.6/100

    Scholarships: Zhiyuan Honor Scholarships
    
    Highly-Graded Major courses:
    \begin{enumerate}
        \item Linear Algebra (Honor): A+ (98)
        \item Fundamental of Programming (Honor): A+ (98)
        \item College Physics: A+ (96)
        \item Introduction to Artificial Intelligence: A+ (95)
    \end{enumerate}

    \textbf{16} courses achieved A/A+, including \textbf{all major-related courses}, with \textbf{8} courses earning A+.
    


  \cvheading
    {High school affiliated to Nanjing Normal University}{Nanjing, China}
    {}{09/2021 - 06/2024}
\cvheadingend

\vspace{-5pt}


\section{Technical Projects}

{\bfseries\selectfont Data\_Structure\_Awesome} (C++)

\url{https://github.com/xiyuanyang-code/Data-Structure-Awesome}

C++ implementations of various data structures and classic algorithms. As the final project for \textbf{CS0501H} (with ACM Honor Class students), I implemented several STL containers, including \texttt{std::vector}, \texttt{std::list}, \texttt{std::priority\_queue}, \texttt{std::linked\_hashmap} and \texttt{std::map}, with the total repository code exceeding \textbf{12,000 line}s.
\newline

% todo: to be finished in the future
% \textbf{RAG Blog Content Retrieval} (Python)

% \url{https://github.com/xiyuanyang-code/RAG_blog_content_retrieval}

% Using \textbf{RAG (Retrieval-Augmented Generation)}, I implemented a system to crawl and retrieve content from specific blogs, along with a simple web UI frontend, enabling LLMs to efficiently search and access the target blog’s content.
% \newline

\textbf{Camel multi-agent English Essay Revision helper} (Python)
\url{https://github.com/xiyuanyang-code/Camel-Agent-English-Essay-Revision} 

As an enthusiast of multi-agent systems, I implemented a collaborative blog post revision system using Camel-AI's multi-agent framework. In practical use, this system outperforms simple prompt engineering with a single LLM.
\newline

\textbf{Vosk-Based Voice Translation} (Python)

\url{https://github.com/xiyuanyang-code/voice_translation}

A Vosk-based Chinese-English bilingual speech recognition Python Application.
\newline

\textbf{My Technical Blog} (HTML)

\url{https://xiyuanyang-code.github.io} 

Maintainer of \href{https://xiyuanyang-code.github.io}{my technical blog}. I regularly publish technical content focusing on computer science and AI. To \href{https://xiyuanyang-code.github.io/Blog-word-counting/}{date}, I have authored \textbf{over 100 articles with a cumulative word count exceeding 340,000 words}.
\newline

I am always an active committer on \href{https://github.com/xiyuanyang-code}{GitHub} with more than 20 open-sourced repositories. I'm passionate about engineering projects and a strong advocate for open-source philosophy.

\section{Skills}
\textbf{Programming Languages}: Python (Proficient), C++ (Proficient), C, HTML, CSS, JavaScript, Rust.

\textbf{Tools}: Git, LaTeX, Shell, Docker.

\textbf{Python Modules}: Torch (Proficient), Numpy, Pandas, Matplotlib.

\textbf{ML \& DL}: Familiar with Deep Learning architectures and model training techniques.

\textbf{Languages}: Chinese (native), English (fluent, CET-4: 611).


\section{Research}

My research interests lie in \textbf{Multi-Agent Systems} and \textbf{LLM Reasoning}. 

\textbf{Undergraduate Research Assistant}, MediaBrain, SJTU, supervised by Siheng Chen.


% ! remove this, this is unnecessary in resume pages
% \subsection{Extracurricular}

% Several landmark courses learned after class, including \textbf{Introduction to Algorithms} (MIT 6.006), \textbf{Fundamentals of Machine Learning}, \textbf{Large Language Model Agents} (Berkeley, CS294/194-196), etc. 

% Through these, I've developed foundational knowledge in traditional machine learning, deep neural networks, and agent architectures.



\section{Awards and Certifications}



\twopartheader{\textbf{Zhiyuan Honor Scholarships}}{2024}
\newline

\twopartheader{\textbf{Honorable Mention in MCM}}{2025}


Honorable Mention, COMAP Mathematical Contest in Modeling / Interdisciplinary Contest in Modeling (MCM/ICM).


2025-C: Models for Olympic Medal Tables.

% \section{Misc.}
% \cvheadingstart
% \item
% \cvitemstart
%   \cvitem{Licenses \& certifications}
%   \cvitem{Volunteering}
%   \cvitem{\href{https://arxiv.org/}{Publications}}
%   \cvitem{\href{https://patents.google.com/}{Patents}}
% \cvitemend
% \cvheadingend

\end{document}