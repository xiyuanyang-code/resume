%------------------------
% Ethan's Résumé Template
% Author: necusjz
% License: CC-BY-4.0
%------------------------
\documentclass[letterpaper,11pt]{article}

\usepackage{latexsym}
\usepackage[empty]{fullpage}
\usepackage{titlesec}
\usepackage{bm}
\usepackage{marvosym}
\usepackage[usenames,dvipsnames]{color}
\usepackage{verbatim}
\usepackage{enumitem}
\usepackage[hidelinks]{hyperref}
\usepackage{fancyhdr}
\usepackage[english]{babel}
\usepackage{tabularx}
\usepackage{fontawesome5}
\usepackage{ragged2e}
\usepackage{etoolbox}
\usepackage{tikz}
\usepackage{hyperref}
\definecolor{lightblue}{RGB}{173, 216, 230} 
\hypersetup{
    colorlinks = true,  
    urlcolor   = blue,  
}
\input{glyphtounicode}

% font options
\usepackage{newpxtext}
\linespread{1.05}  
\usepackage[T1]{fontenc}

\pagestyle{fancy}
\fancyhf{}
\fancyfoot{}
\renewcommand{\headrulewidth}{0pt}
\renewcommand{\footrulewidth}{0pt}

% adjust margins
\addtolength{\oddsidemargin}{-0.5in}
\addtolength{\evensidemargin}{-0.5in}
\addtolength{\textwidth}{1in}
\addtolength{\topmargin}{-.5in}
\addtolength{\textheight}{1.0in}

\urlstyle{same}

\raggedbottom
\raggedright
\setlength{\tabcolsep}{0in}
\setlength{\footskip}{5pt}

% sections formatting
\titleformat{\section}{
  \vspace{-2pt}\scshape\raggedright\large
}{}{0em}{}[\color{black}\titlerule\vspace{-5pt}]


% subsection formatting
\titleformat{\subsection}{
  \vspace{-2pt}\scshape\raggedright\normalsize
}{}{0em}{}[\color{black}\titlerule\vspace{-5pt}]
% ensure that generate pdf is machine readable/ATS parsable
\pdfgentounicode=1

% custom commands
\newcommand{\cvitem}[1]{
  \item\small{
    {#1\vspace{-2pt}}
  }
}

\newcommand{\cvheading}[4]{
  \vspace{-2pt}\item
    \begin{tabular*}{\textwidth}[t]{l@{\extracolsep{\fill}}r}
      \textbf{#1} & #2 \\
      \small#3 & \small #4 \\
    \end{tabular*}\vspace{-7pt}
}

\newcommand{\twopartheader}[2]{
  \vspace{-2pt}\noindent
  \begin{tabular*}{\textwidth}{l@{\extracolsep{\fill}}r}
    #1 & #2 \\
  \end{tabular*}\vspace{-7pt}
}

\newcommand{\cvheadingstart}{\begin{itemize}[leftmargin=0in, label={}]}
\newcommand{\cvheadingend}{\end{itemize}}
\newcommand{\cvitemstart}{\begin{itemize}[label=\textopenbullet]\justifying}
\newcommand{\cvitemend}{\end{itemize}\vspace{-5pt}}

\newcommand{\cvskill}[2]{
  \textcolor{black}{\textbf{#1}}\hfill
  \foreach \x in {1,...,5}{%
    \space{\ifnumgreater{\x}{#2}{\color{black!80!white!20}}{\color{black}}\faSquare}}\par%
  \vspace{-2pt}
}

\renewcommand\labelitemii{$\vcenter{\hbox{\footnotesize$\bullet$}}$}


% =======================begin==============================
\begin{document}

% contact information
\begin{center}
  \textbf{\LARGE\scshape Xiyuan Yang} \\
  \vspace{1pt}\small
  \href{mailto:yangxiyuan@sjtu.edu.cn}{yangxiyuan@sjtu.edu.cn}
  $\ \diamond\ $
  Shanghai Jiao Tong University
  $\ \diamond\ $
  Shanghai, China
  \\
  GitHub: \href{https://github.com/xiyuanyang-code}{https://github.com/xiyuanyang-code}
  \\
  Resume: \href{https://xiyuanyang-code.github.io/resume}{https://xiyuanyang-code.github.io/resume/}
\end{center}



\section{Education}
\cvheadingstart
\cvheading
{Shanghai Jiao Tong University}{Shanghai, China}
{Bachelor's degree, School of artificial intelligence}{09/2024 - present}

GPA: 4.14/4.3 (Ranked \textbf{1} out of 62)

Score: 94.0/100

Scholarships: Zhiyuan Honor Scholarships (first 5\%)

\begin{itemize}[nosep]
  \item Comprehensive Programming Practice: 100/100
  \item Linear Algebra (Honor): 98/100
  \item Fundamental of Programming (Honor): 98/100
        % added courses should be less than 4 courses, all 98+
        % \item College Physics I (Honor): 96/100
        % \item Introduction to Artificial Intelligence: 95/100
        % \item Mathematical Analysis II (Honor): 95/100
\end{itemize}
\textbf{17} courses achieved A/A+, with \textbf{9} of them earning A+. This includes \textbf{all major-related courses}.
\cvheading
{High school affiliated to Nanjing Normal University}{Nanjing, China}
{}{09/2021 - 06/2024}
\cvheadingend
\vspace{-5pt}




\section{Technical Projects}

% ============== Project1 AlphaBuild ============== %
\textbf{AlphaBuild: Generating Formulaic Alphas on a Wider Range of Stock Data}

Leading AlphaBuild as the final project of course \textbf{AI1803}, an RL-based methodology using PPO to optimize factor mining process based on \href{https://github.com/RL-MLDM/alphagen}{AlphaGen}, achieving superior backtest performance by leveraging smaller, parallel factor pools and PCA for final factor selection.
\newline

% =================================================== %


% ============== Project2 Data Structure Awesome ============== %
\textbf{Data Structure Awesome} (C++)

\url{https://github.com/xiyuanyang-code/Data-Structure-Awesome}

C++ implementations of various data structures and classic algorithms. As the final project for \textbf{CS0501H} (with ACM Honor Class students), I implemented several STL containers, including \texttt{std::vector}, \texttt{std::list}, \texttt{std::priority\_queue}, \texttt{std::linked\_hashmap} and \texttt{std::map}.
\newline

% =================================================== %


% todo finish several projects in the future
% Bloging AI (Maybe in the future)
% ProbeCode


\textbf{My Technical Blog}

\url{https://xiyuanyang-code.github.io}

Maintainer of \href{https://xiyuanyang-code.github.io}{my technical blog}. I regularly publish technical content focusing on computer science and AI. To \href{https://xiyuanyang-code.github.io/Blog-word-counting/}{date}, I have authored \textbf{over 120 articles with a cumulative word count exceeding 400,000 words}.

Active GitHub committer (\href{https://github.com/xiyuanyang-code}{xiyuanyang-code}) with 30+ open-source repositories. I build and open-source user-friendly tools I personally find useful (See \href{https://xiyuanyang-code.github.io/Tool-Zoo/}{Tool Zoo}), driven by a passion for engineering and open source.

\section{Skills}
\textbf{Programming Languages}: Python (Proficient), C++ (Proficient), Rust, JavaScript, HTML, CSS.

\textbf{Tools}: Git, LaTeX, Shell, Docker.

\textbf{Python Modules}: Torch (Proficient), Numpy, Pandas, Matplotlib, BeautifulSoup.

\textbf{ML \& DL}: Familiar with Deep Learning architectures and model training techniques.

\textbf{Languages}: Chinese (native), English (CET-6: 584).


\section{Research}

My research interests lie in \textbf{Agentic AI} and  \textbf{Multi-Agent Systems}.
\newline

\textbf{AppCopilot: Toward General, Accurate, Long‑Horizon, and Efficient Mobile Agent}
\url{https://arxiv.org/abs/2509.02444}

Developed \textbf{AppCopilot}, a multimodal, multi-agent on-device assistant that addresses core challenges in the mobile-agent landscape. AppCopilot significantly improves generalization, on-screen interaction accuracy, and long-horizon task completion, while also optimizing performance and efficiency on resource-constrained mobile devices.
\newline


\section{Awards and Certifications}

\twopartheader{\textbf{Zhiyuan Honor Scholarships} (first 5\%)}{2024}
\newline

\twopartheader{\textbf{Honorable Mention in MCM}}{2025}

Honorable Mention, COMAP Mathematical Contest in Modeling / Interdisciplinary Contest in Modeling (MCM/ICM).

2025-C: Models for Olympic Medal Tables.

\end{document}